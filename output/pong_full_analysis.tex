
\documentclass[12pt, a4paper]{article}
\usepackage[utf8]{inputenc}
\usepackage{amsmath}
\usepackage{geometry}
\geometry{a4paper, margin=1in}
\usepackage{lmodern}
\usepackage{amssymb}
\usepackage{graphicx}
\usepackage{listings}
\usepackage{xcolor}

\definecolor{codegreen}{rgb}{0,0.6,0}
\definecolor{codegray}{rgb}{0.5,0.5,0.5}
\definecolor{codepurple}{rgb}{0.58,0,0.82}
\definecolor{backcolour}{rgb}{0.95,0.95,0.92}

\lstdefinestyle{mystyle}{
    backgroundcolor=\color{backcolour},   
    commentstyle=\color{codegreen},
    keywordstyle=\color{magenta},
    numberstyle=\tiny\color{codegray},
    stringstyle=\color{codepurple},
    basicstyle=\ttfamily\footnotesize,
    breakatwhitespace=false,         
    breaklines=true,                 
    captionpos=b,                    
    keepspaces=true,                 
    numbers=left,                    
    numbersep=5pt,                  
    showspaces=false,                
    showstringspaces=false,
    showtabs=false,                  
    tabsize=2
}
\lstset{style=mystyle}

\allowdisplaybreaks

\title{Análisis Matemático de un Programa en C \\ \large Generado por Project Diophantus}
\author{}
\date{\today}

\begin{document}
\maketitle

\section*{Resumen Ejecutivo}
Este documento presenta la traducción completa de un programa de software escrito en C a un objeto matemático puro: un sistema de ecuaciones diofánticas. El proceso demuestra la equivalencia fundamental entre la computación (algoritmos) y la teoría de números (polinomios), como lo postula el Teorema de Matiyasevich (MRDP).

El compilador ha analizado el código fuente y ha extraído las siguientes componentes clave para describir una única transición de estado (un "fotograma"):
\begin{itemize}
    \item \textbf{Variables de Estado ($S_t$):} Las variables que definen el estado del sistema en un instante $t$. Para este programa, son: b, c, d, e, f, g, p, q.
    \item \textbf{Variables de Entrada ($I_t$):} Las variables que representan la interacción con el exterior en el instante $t$. Para este programa, son: getch, kbhit.
\end{itemize}

El documento se divide en dos partes principales, que corresponden a las dos grandes fases de la traducción:
\begin{enumerate}
    \item \textbf{La Función de Transición de Estado:} Muestra cómo la lógica procedural del programa (bucles, condicionales) se "aplana" en un sistema de ecuaciones de asignación $S_{t+1} = F(S_t, I_t)$, y cómo este sistema se optimiza para revelar su estructura.
    \item \textbf{La Conversión a Polinomio Puro:} Muestra cómo el sistema de asignación, que aún contiene operadores lógicos, se convierte en un sistema de ecuaciones diofánticas puras (solo usando suma, resta y multiplicación), cumpliendo con el objetivo teórico final del proyecto.
\end{enumerate}

\part{La Función de Transición de Estado}
\section{Aplanamiento y Optimización}
El primer paso consiste en convertir la lógica imperativa del bucle principal del programa en una función matemática estática, $F$. Esto se logra mediante un proceso de "aplanamiento" que transforma construcciones como `if-else` en expresiones aritméticas y sustituye todas las variables temporales hasta que cada ecuación solo dependa del estado anterior ($S_t$) y las entradas ($I_t$).

\subsection*{Ecuaciones de Estado (Forma Pura, Sin Optimizar)}
Esta es la forma "pura" de la función de transición. Cada ecuación es matemáticamente autocontenida y muestra la dependencia total del estado anterior. Su complejidad y repetición visual reflejan la necesidad de optimización.
\begin{align*}
b[t+1] &= \parbox[t]{0.8\linewidth}{(((b + d) < 1) \cdot (40) + (1 - ((b + d) < 1)) \cdot ((((b + d) > 78) \cdot (40) + (1 - ((b + d) > 78)) \cdot ((b + d)))))}\\
c[t+1] &= \parbox[t]{0.8\linewidth}{(((b + d) < 1) \cdot (12) + (1 - ((b + d) < 1)) \cdot ((((b + d) > 78) \cdot (12) + (1 - ((b + d) > 78)) \cdot ((c + e)))))}\\
d[t+1] &= \parbox[t]{0.8\linewidth}{((((((b + d) = 2) \land ((c + e) \geq ((((kbhit \cdot (getch) + (1 - kbhit) \cdot (0)) = 119) \land (p > 1)) \cdot ((p - 1)) + (1 - (((kbhit \cdot (getch) + (1 - kbhit) \cdot (0)) = 119) \land (p > 1))) \cdot (((((kbhit \cdot (getch) + (1 - kbhit) \cdot (0)) = 115) \land (p < 18)) \cdot ((p + 1)) + (1 - (((kbhit \cdot (getch) + (1 - kbhit) \cdot (0)) = 115) \land (p < 18))) \cdot (p)))))) \land ((c + e) < (((((kbhit \cdot (getch) + (1 - kbhit) \cdot (0)) = 119) \land (p > 1)) \cdot ((p - 1)) + (1 - (((kbhit \cdot (getch) + (1 - kbhit) \cdot (0)) = 119) \land (p > 1))) \cdot (((((kbhit \cdot (getch) + (1 - kbhit) \cdot (0)) = 115) \land (p < 18)) \cdot ((p + 1)) + (1 - (((kbhit \cdot (getch) + (1 - kbhit) \cdot (0)) = 115) \land (p < 18))) \cdot (p)))) + 5))) \lor ((((b + d) = 77) \land ((c + e) \geq ((((kbhit \cdot (getch) + (1 - kbhit) \cdot (0)) = 105) \land (q > 1)) \cdot ((q - 1)) + (1 - (((kbhit \cdot (getch) + (1 - kbhit) \cdot (0)) = 105) \land (q > 1))) \cdot (((((kbhit \cdot (getch) + (1 - kbhit) \cdot (0)) = 107) \land (q < 18)) \cdot ((q + 1)) + (1 - (((kbhit \cdot (getch) + (1 - kbhit) \cdot (0)) = 107) \land (q < 18))) \cdot (q)))))) \land ((c + e) < (((((kbhit \cdot (getch) + (1 - kbhit) \cdot (0)) = 105) \land (q > 1)) \cdot ((q - 1)) + (1 - (((kbhit \cdot (getch) + (1 - kbhit) \cdot (0)) = 105) \land (q > 1))) \cdot (((((kbhit \cdot (getch) + (1 - kbhit) \cdot (0)) = 107) \land (q < 18)) \cdot ((q + 1)) + (1 - (((kbhit \cdot (getch) + (1 - kbhit) \cdot (0)) = 107) \land (q < 18))) \cdot (q)))) + 5)))) \cdot ((0 - d)) + (1 - (((((b + d) = 2) \land ((c + e) \geq ((((kbhit \cdot (getch) + (1 - kbhit) \cdot (0)) = 119) \land (p > 1)) \cdot ((p - 1)) + (1 - (((kbhit \cdot (getch) + (1 - kbhit) \cdot (0)) = 119) \land (p > 1))) \cdot (((((kbhit \cdot (getch) + (1 - kbhit) \cdot (0)) = 115) \land (p < 18)) \cdot ((p + 1)) + (1 - (((kbhit \cdot (getch) + (1 - kbhit) \cdot (0)) = 115) \land (p < 18))) \cdot (p)))))) \land ((c + e) < (((((kbhit \cdot (getch) + (1 - kbhit) \cdot (0)) = 119) \land (p > 1)) \cdot ((p - 1)) + (1 - (((kbhit \cdot (getch) + (1 - kbhit) \cdot (0)) = 119) \land (p > 1))) \cdot (((((kbhit \cdot (getch) + (1 - kbhit) \cdot (0)) = 115) \land (p < 18)) \cdot ((p + 1)) + (1 - (((kbhit \cdot (getch) + (1 - kbhit) \cdot (0)) = 115) \land (p < 18))) \cdot (p)))) + 5))) \lor ((((b + d) = 77) \land ((c + e) \geq ((((kbhit \cdot (getch) + (1 - kbhit) \cdot (0)) = 105) \land (q > 1)) \cdot ((q - 1)) + (1 - (((kbhit \cdot (getch) + (1 - kbhit) \cdot (0)) = 105) \land (q > 1))) \cdot (((((kbhit \cdot (getch) + (1 - kbhit) \cdot (0)) = 107) \land (q < 18)) \cdot ((q + 1)) + (1 - (((kbhit \cdot (getch) + (1 - kbhit) \cdot (0)) = 107) \land (q < 18))) \cdot (q)))))) \land ((c + e) < (((((kbhit \cdot (getch) + (1 - kbhit) \cdot (0)) = 105) \land (q > 1)) \cdot ((q - 1)) + (1 - (((kbhit \cdot (getch) + (1 - kbhit) \cdot (0)) = 105) \land (q > 1))) \cdot (((((kbhit \cdot (getch) + (1 - kbhit) \cdot (0)) = 107) \land (q < 18)) \cdot ((q + 1)) + (1 - (((kbhit \cdot (getch) + (1 - kbhit) \cdot (0)) = 107) \land (q < 18))) \cdot (q)))) + 5))))) \cdot (d))}\\
e[t+1] &= ((((c + e) = 1) \lor ((c + e) = 22)) \cdot ((0 - e)) + (1 - (((c + e) = 1) \lor ((c + e) = 22))) \cdot (e))\\
f[t+1] &= (((b + d) < 1) \cdot ((f + 1)) + (1 - ((b + d) < 1)) \cdot (f))\\
g[t+1] &= \parbox[t]{0.8\linewidth}{(((b + d) < 1) \cdot (g) + (1 - ((b + d) < 1)) \cdot ((((b + d) > 78) \cdot ((g + 1)) + (1 - ((b + d) > 78)) \cdot (g))))}\\
p[t+1] &= \parbox[t]{0.8\linewidth}{((((kbhit \cdot (getch) + (1 - kbhit) \cdot (0)) = 119) \land (p > 1)) \cdot ((p - 1)) + (1 - (((kbhit \cdot (getch) + (1 - kbhit) \cdot (0)) = 119) \land (p > 1))) \cdot (((((kbhit \cdot (getch) + (1 - kbhit) \cdot (0)) = 115) \land (p < 18)) \cdot ((p + 1)) + (1 - (((kbhit \cdot (getch) + (1 - kbhit) \cdot (0)) = 115) \land (p < 18))) \cdot (p))))}\\
q[t+1] &= \parbox[t]{0.8\linewidth}{((((kbhit \cdot (getch) + (1 - kbhit) \cdot (0)) = 105) \land (q > 1)) \cdot ((q - 1)) + (1 - (((kbhit \cdot (getch) + (1 - kbhit) \cdot (0)) = 105) \land (q > 1))) \cdot (((((kbhit \cdot (getch) + (1 - kbhit) \cdot (0)) = 107) \land (q < 18)) \cdot ((q + 1)) + (1 - (((kbhit \cdot (getch) + (1 - kbhit) \cdot (0)) = 107) \land (q < 18))) \cdot (q))))}
\end{align*}

\subsection*{Definiciones de Cálculos Comunes (CSE)}
Para simplificar y hacer las ecuaciones manejables, el sistema busca expresiones que se repiten (ej. la lógica de movimiento de una pala), les asigna un nombre simbólico (ej. $C_0, C_1, \dots$) y las calcula una sola vez. Estas definiciones representan los bloques de construcción lógicos del programa.
\begin{align*}
C_0 &= (C_1 < 1) \\
C_1 &= (b + d) \\
C_2 &= (C_1 > 78) \\
C_3 &= (C_4 \cdot C_8 + (1 - C_4) \cdot C_9) \\
C_4 &= (C_5 \land C_7) \\
C_5 &= (C_6 = 105) \\
C_6 &= (kbhit \cdot getch + (1 - kbhit) \cdot 0) \\
C_7 &= (q > 1) \\
C_8 &= (q - 1) \\
C_9 &= (C_10 \cdot C_13 + (1 - C_10) \cdot q) \\
C_10 &= (C_11 \land C_12) \\
C_11 &= (C_6 = 107) \\
C_12 &= (q < 18) \\
C_13 &= (q + 1) \\
C_14 &= (C_15 \cdot C_18 + (1 - C_15) \cdot C_19) \\
C_15 &= (C_16 \land C_17) \\
C_16 &= (C_6 = 119) \\
C_17 &= (p > 1) \\
C_18 &= (p - 1) \\
C_19 &= (C_20 \cdot C_23 + (1 - C_20) \cdot p) \\
C_20 &= (C_21 \land C_22) \\
C_21 &= (C_6 = 115) \\
C_22 &= (p < 18) \\
C_23 &= (p + 1) \\
C_24 &= (c + e)
\end{align*}

\subsection*{Ecuaciones de Estado Finales (Optimizadas con CSE)}
Esta es la versión final y simplificada de la función de transición. Utiliza las definiciones de $C_n$ para ser más compacta, legible y eficiente. Esta forma es la que más se asemeja a cómo un humano estructuraría los cálculos.
\begin{align*}
b[t+1] &= (C_0 \cdot 40 + (1 - C_0) \cdot (C_2 \cdot 40 + (1 - C_2) \cdot C_1))\\
c[t+1] &= (C_0 \cdot 12 + (1 - C_0) \cdot (C_2 \cdot 12 + (1 - C_2) \cdot C_24))\\
d[t+1] &= \parbox[t]{0.8\linewidth}{(((((C_1 = 2) \land (C_24 \geq C_14)) \land (C_24 < (C_14 + 5))) \lor (((C_1 = 77) \land (C_24 \geq C_3)) \land (C_24 < (C_3 + 5)))) \cdot (0 - d) + (1 - ((((C_1 = 2) \land (C_24 \geq C_14)) \land (C_24 < (C_14 + 5))) \lor (((C_1 = 77) \land (C_24 \geq C_3)) \land (C_24 < (C_3 + 5))))) \cdot d)}\\
e[t+1] &= (((C_24 = 1) \lor (C_24 = 22)) \cdot (0 - e) + (1 - ((C_24 = 1) \lor (C_24 = 22))) \cdot e)\\
f[t+1] &= (C_0 \cdot (f + 1) + (1 - C_0) \cdot f)\\
g[t+1] &= (C_0 \cdot g + (1 - C_0) \cdot (C_2 \cdot (g + 1) + (1 - C_2) \cdot g))\\
p[t+1] &= C_14\\
q[t+1] &= C_3
\end{align*}

\part{Conversión a Polinomio Puro}
\section{Traducción a Ecuaciones Diofánticas}
El paso final y más profundo es convertir la función de transición (que aún contiene operadores lógicos como `==`, `<`, etc.) en un sistema que solo utiliza aritmética entera (suma, resta, multiplicación). Esto se logra introduciendo variables existenciales ($e_n$) y aplicando trucos de la teoría de números, como el Teorema de los Cuatro Cuadrados de Lagrange para manejar las desigualdades.

El proceso ha introducido \textbf{134 variables existenciales} para producir un sistema de \textbf{102 ecuaciones puras}.

\subsection*{Sistema de Ecuaciones Diofánticas Puras (Forma Práctica)}
Esta es la representación más útil para aplicaciones de ingeniería, como la simulación o la síntesis de hardware. Es un sistema de ecuaciones interdependientes que deben satisfacerse simultáneamente. Cada línea representa un cálculo simple o una restricción lógica.
\begin{align*}
e_{1} - (1 - 1) &= 0\\
C_{0}  \cdot  (1 - C_{0}) &= 0\\
C_{0}  \cdot  ((e_{1}) - (C_{1}) - (e_{2}^{2} + e_{3}^{2} + e_{4}^{2} + e_{5}^{2})) &= 0\\
(1 - C_{0})  \cdot  ((C_{1}) - (e_{1}) - 1 - (e_{6}^{2} + e_{7}^{2} + e_{8}^{2} + e_{9}^{2})) &= 0\\
C_{1} - (b + d) &= 0\\
e_{12} - (C_{1} - 1) &= 0\\
C_{2}  \cdot  (1 - C_{2}) &= 0\\
C_{2}  \cdot  ((e_{12}) - (78) - (e_{13}^{2} + e_{14}^{2} + e_{15}^{2} + e_{16}^{2})) &= 0\\
(1 - C_{2})  \cdot  ((78) - (e_{12}) - 1 - (e_{17}^{2} + e_{18}^{2} + e_{19}^{2} + e_{20}^{2})) &= 0\\
C_{3} - ((C_{4})  \cdot  (C_{8}) + (1 - C_{4})  \cdot  (C_{9})) &= 0\\
C_{4} - (C_{5}  \cdot  C_{7}) &= 0\\
C_{5}  \cdot  (1 - C_{5}) &= 0\\
C_{5}  \cdot  ((C_{6}) - (105)) &= 0\\
((C_{6}) - (105))  \cdot  e_{21} - (1 - C_{5}) &= 0\\
C_{6} - ((kbhit)  \cdot  (getch) + (1 - kbhit)  \cdot  (0)) &= 0\\
e_{24} - (q - 1) &= 0\\
C_{7}  \cdot  (1 - C_{7}) &= 0\\
C_{7}  \cdot  ((e_{24}) - (1) - (e_{25}^{2} + e_{26}^{2} + e_{27}^{2} + e_{28}^{2})) &= 0\\
(1 - C_{7})  \cdot  ((1) - (e_{24}) - 1 - (e_{29}^{2} + e_{30}^{2} + e_{31}^{2} + e_{32}^{2})) &= 0\\
C_{8} - (q - 1) &= 0\\
C_{9} - ((C_{10})  \cdot  (C_{13}) + (1 - C_{10})  \cdot  (q)) &= 0\\
C_{10} - (C_{11}  \cdot  C_{12}) &= 0\\
C_{11}  \cdot  (1 - C_{11}) &= 0\\
C_{11}  \cdot  ((C_{6}) - (107)) &= 0\\
((C_{6}) - (107))  \cdot  e_{33} - (1 - C_{11}) &= 0\\
e_{35} - (18 - 1) &= 0\\
C_{12}  \cdot  (1 - C_{12}) &= 0\\
C_{12}  \cdot  ((e_{35}) - (q) - (e_{36}^{2} + e_{37}^{2} + e_{38}^{2} + e_{39}^{2})) &= 0\\
(1 - C_{12})  \cdot  ((q) - (e_{35}) - 1 - (e_{40}^{2} + e_{41}^{2} + e_{42}^{2} + e_{43}^{2})) &= 0\\
C_{13} - (q + 1) &= 0\\
C_{14} - ((C_{15})  \cdot  (C_{18}) + (1 - C_{15})  \cdot  (C_{19})) &= 0\\
C_{15} - (C_{16}  \cdot  C_{17}) &= 0\\
C_{16}  \cdot  (1 - C_{16}) &= 0\\
C_{16}  \cdot  ((C_{6}) - (119)) &= 0\\
((C_{6}) - (119))  \cdot  e_{44} - (1 - C_{16}) &= 0\\
e_{47} - (p - 1) &= 0\\
C_{17}  \cdot  (1 - C_{17}) &= 0\\
C_{17}  \cdot  ((e_{47}) - (1) - (e_{48}^{2} + e_{49}^{2} + e_{50}^{2} + e_{51}^{2})) &= 0\\
(1 - C_{17})  \cdot  ((1) - (e_{47}) - 1 - (e_{52}^{2} + e_{53}^{2} + e_{54}^{2} + e_{55}^{2})) &= 0\\
C_{18} - (p - 1) &= 0\\
C_{19} - ((C_{20})  \cdot  (C_{23}) + (1 - C_{20})  \cdot  (p)) &= 0\\
C_{20} - (C_{21}  \cdot  C_{22}) &= 0\\
C_{21}  \cdot  (1 - C_{21}) &= 0\\
C_{21}  \cdot  ((C_{6}) - (115)) &= 0\\
((C_{6}) - (115))  \cdot  e_{56} - (1 - C_{21}) &= 0\\
e_{58} - (18 - 1) &= 0\\
C_{22}  \cdot  (1 - C_{22}) &= 0\\
C_{22}  \cdot  ((e_{58}) - (p) - (e_{59}^{2} + e_{60}^{2} + e_{61}^{2} + e_{62}^{2})) &= 0\\
(1 - C_{22})  \cdot  ((p) - (e_{58}) - 1 - (e_{63}^{2} + e_{64}^{2} + e_{65}^{2} + e_{66}^{2})) &= 0\\
C_{23} - (p + 1) &= 0\\
C_{24} - (c + e) &= 0\\
e_{67} - ((C_{2})  \cdot  (40) + (1 - C_{2})  \cdot  (C_{1})) &= 0\\
b[t+1] - ((C_{0})  \cdot  (40) + (1 - C_{0})  \cdot  (e_{67})) &= 0\\
e_{68} - ((C_{2})  \cdot  (12) + (1 - C_{2})  \cdot  (C_{24})) &= 0\\
c[t+1] - ((C_{0})  \cdot  (12) + (1 - C_{0})  \cdot  (e_{68})) &= 0\\
e_{72}  \cdot  (1 - e_{72}) &= 0\\
e_{72}  \cdot  ((C_{1}) - (2)) &= 0\\
((C_{1}) - (2))  \cdot  e_{73} - (1 - e_{72}) &= 0\\
e_{74}  \cdot  (1 - e_{74}) &= 0\\
e_{74}  \cdot  ((C_{24}) - (C_{14}) - (e_{76}^{2} + e_{77}^{2} + e_{78}^{2} + e_{79}^{2})) &= 0\\
(1 - e_{74})  \cdot  ((C_{14}) - (C_{24}) - 1 - (e_{80}^{2} + e_{81}^{2} + e_{82}^{2} + e_{83}^{2})) &= 0\\
e_{71} - (e_{72}  \cdot  e_{74}) &= 0\\
e_{85} - (C_{14} + 5) &= 0\\
e_{88} - (C_{14} + 5) &= 0\\
e_{87} - (e_{88} - 1) &= 0\\
e_{84}  \cdot  (1 - e_{84}) &= 0\\
e_{84}  \cdot  ((e_{87}) - (C_{24}) - (e_{89}^{2} + e_{90}^{2} + e_{91}^{2} + e_{92}^{2})) &= 0\\
(1 - e_{84})  \cdot  ((C_{24}) - (e_{87}) - 1 - (e_{93}^{2} + e_{94}^{2} + e_{95}^{2} + e_{96}^{2})) &= 0\\
e_{70} - (e_{71}  \cdot  e_{84}) &= 0\\
e_{99}  \cdot  (1 - e_{99}) &= 0\\
e_{99}  \cdot  ((C_{1}) - (77)) &= 0\\
((C_{1}) - (77))  \cdot  e_{100} - (1 - e_{99}) &= 0\\
e_{101}  \cdot  (1 - e_{101}) &= 0\\
e_{101}  \cdot  ((C_{24}) - (C_{3}) - (e_{103}^{2} + e_{104}^{2} + e_{105}^{2} + e_{106}^{2})) &= 0\\
(1 - e_{101})  \cdot  ((C_{3}) - (C_{24}) - 1 - (e_{107}^{2} + e_{108}^{2} + e_{109}^{2} + e_{110}^{2})) &= 0\\
e_{98} - (e_{99}  \cdot  e_{101}) &= 0\\
e_{112} - (C_{3} + 5) &= 0\\
e_{115} - (C_{3} + 5) &= 0\\
e_{114} - (e_{115} - 1) &= 0\\
e_{111}  \cdot  (1 - e_{111}) &= 0\\
e_{111}  \cdot  ((e_{114}) - (C_{24}) - (e_{116}^{2} + e_{117}^{2} + e_{118}^{2} + e_{119}^{2})) &= 0\\
(1 - e_{111})  \cdot  ((C_{24}) - (e_{114}) - 1 - (e_{120}^{2} + e_{121}^{2} + e_{122}^{2} + e_{123}^{2})) &= 0\\
e_{97} - (e_{98}  \cdot  e_{111}) &= 0\\
e_{69} - (e_{70} + e_{97} - e_{70}  \cdot  e_{97}) &= 0\\
e_{124} - (0 - d) &= 0\\
d[t+1] - ((e_{69})  \cdot  (e_{124}) + (1 - e_{69})  \cdot  (d)) &= 0\\
e_{126}  \cdot  (1 - e_{126}) &= 0\\
e_{126}  \cdot  ((C_{24}) - (1)) &= 0\\
((C_{24}) - (1))  \cdot  e_{127} - (1 - e_{126}) &= 0\\
e_{128}  \cdot  (1 - e_{128}) &= 0\\
e_{128}  \cdot  ((C_{24}) - (22)) &= 0\\
((C_{24}) - (22))  \cdot  e_{129} - (1 - e_{128}) &= 0\\
e_{125} - (e_{126} + e_{128} - e_{126}  \cdot  e_{128}) &= 0\\
e_{130} - (0 - e) &= 0\\
e[t+1] - ((e_{125})  \cdot  (e_{130}) + (1 - e_{125})  \cdot  (e)) &= 0\\
e_{131} - (f + 1) &= 0\\
f[t+1] - ((C_{0})  \cdot  (e_{131}) + (1 - C_{0})  \cdot  (f)) &= 0\\
e_{133} - (g + 1) &= 0\\
e_{132} - ((C_{2})  \cdot  (e_{133}) + (1 - C_{2})  \cdot  (g)) &= 0\\
g[t+1] - ((C_{0})  \cdot  (g) + (1 - C_{0})  \cdot  (e_{132})) &= 0\\
p[t+1] - (C_{14}) &= 0\\
q[t+1] - (C_{3}) &= 0
\end{align*}
\subsection*{Ecuación Polinómica Única (Forma Teórica P=0)}
Por completitud teórica, el sistema anterior puede ser combinado en una única ecuación mediante la suma de los cuadrados de cada ecuación. Una solución entera a esta única y masiva ecuación corresponde a una transición de estado válida del programa original. Esta es la forma final que demuestra el Teorema MRDP.
\begin{align*}
& (e_{1} - (1 - 1))^{2} + (C_{0}  \cdot  (1 - C_{0}))^{2} \\
& + (C_{0}  \cdot  ((e_{1}) - (C_{1}) - (e_{2}^{2} + e_{3}^{2} + e_{4}^{2} + e_{5}^{2})))^{2} \\
& + ((1 - C_{0})  \cdot  ((C_{1}) - (e_{1}) - 1 - (e_{6}^{2} + e_{7}^{2} + e_{8}^{2} \\
& + e_{9}^{2})))^{2} + (C_{1} - (b + d))^{2} + (e_{12} - (C_{1} - 1))^{2} \\
& + (C_{2}  \cdot  (1 - C_{2}))^{2} + (C_{2}  \cdot  ((e_{12}) - (78) - (e_{13}^{2} \\
& + e_{14}^{2} + e_{15}^{2} + e_{16}^{2})))^{2} \\
& + ((1 - C_{2})  \cdot  ((78) - (e_{12}) - 1 - (e_{17}^{2} + e_{18}^{2} + e_{19}^{2} \\
& + e_{20}^{2})))^{2} + (C_{3} - ((C_{4})  \cdot  (C_{8}) + (1 - C_{4})  \cdot  (C_{9})))^{2} \\
& + (C_{4} - (C_{5}  \cdot  C_{7}))^{2} + (C_{5}  \cdot  (1 - C_{5}))^{2} \\
& + (C_{5}  \cdot  ((C_{6}) - (105)))^{2} \\
& + (((C_{6}) - (105))  \cdot  e_{21} - (1 - C_{5}))^{2} + (C_{6} - ((kbhit)  \cdot  (getch) \\
& + (1 - kbhit)  \cdot  (0)))^{2} + (e_{24} - (q - 1))^{2} + (C_{7}  \cdot  (1 - C_{7}))^{2} \\
& + (C_{7}  \cdot  ((e_{24}) - (1) - (e_{25}^{2} + e_{26}^{2} + e_{27}^{2} + e_{28}^{2})))^{2} \\
& + ((1 - C_{7})  \cdot  ((1) - (e_{24}) - 1 - (e_{29}^{2} + e_{30}^{2} + e_{31}^{2} \\
& + e_{32}^{2})))^{2} + (C_{8} - (q - 1))^{2} + (C_{9} - ((C_{10})  \cdot  (C_{13}) \\
& + (1 - C_{10})  \cdot  (q)))^{2} + (C_{10} - (C_{11}  \cdot  C_{12}))^{2} \\
& + (C_{11}  \cdot  (1 - C_{11}))^{2} + (C_{11}  \cdot  ((C_{6}) - (107)))^{2} \\
& + (((C_{6}) - (107))  \cdot  e_{33} - (1 - C_{11}))^{2} + (e_{35} - (18 - 1))^{2} \\
& + (C_{12}  \cdot  (1 - C_{12}))^{2} + (C_{12}  \cdot  ((e_{35}) - (q) - (e_{36}^{2} \\
& + e_{37}^{2} + e_{38}^{2} + e_{39}^{2})))^{2} \\
& + ((1 - C_{12})  \cdot  ((q) - (e_{35}) - 1 - (e_{40}^{2} + e_{41}^{2} + e_{42}^{2} \\
& + e_{43}^{2})))^{2} + (C_{13} - (q + 1))^{2} + (C_{14} - ((C_{15})  \cdot  (C_{18}) \\
& + (1 - C_{15})  \cdot  (C_{19})))^{2} + (C_{15} - (C_{16}  \cdot  C_{17}))^{2} \\
& + (C_{16}  \cdot  (1 - C_{16}))^{2} + (C_{16}  \cdot  ((C_{6}) - (119)))^{2} \\
& + (((C_{6}) - (119))  \cdot  e_{44} - (1 - C_{16}))^{2} + (e_{47} - (p - 1))^{2} \\
& + (C_{17}  \cdot  (1 - C_{17}))^{2} + (C_{17}  \cdot  ((e_{47}) - (1) - (e_{48}^{2} \\
& + e_{49}^{2} + e_{50}^{2} + e_{51}^{2})))^{2} \\
& + ((1 - C_{17})  \cdot  ((1) - (e_{47}) - 1 - (e_{52}^{2} + e_{53}^{2} + e_{54}^{2} \\
& + e_{55}^{2})))^{2} + (C_{18} - (p - 1))^{2} + (C_{19} - ((C_{20})  \cdot  (C_{23}) \\
& + (1 - C_{20})  \cdot  (p)))^{2} + (C_{20} - (C_{21}  \cdot  C_{22}))^{2} \\
& + (C_{21}  \cdot  (1 - C_{21}))^{2} + (C_{21}  \cdot  ((C_{6}) - (115)))^{2} \\
& + (((C_{6}) - (115))  \cdot  e_{56} - (1 - C_{21}))^{2} + (e_{58} - (18 - 1))^{2} \\
& + (C_{22}  \cdot  (1 - C_{22}))^{2} + (C_{22}  \cdot  ((e_{58}) - (p) - (e_{59}^{2} \\
& + e_{60}^{2} + e_{61}^{2} + e_{62}^{2})))^{2} \\
& + ((1 - C_{22})  \cdot  ((p) - (e_{58}) - 1 - (e_{63}^{2} + e_{64}^{2} + e_{65}^{2} \\
& + e_{66}^{2})))^{2} + (C_{23} - (p + 1))^{2} + (C_{24} - (c + e))^{2} \\
& + (e_{67} - ((C_{2})  \cdot  (40) + (1 - C_{2})  \cdot  (C_{1})))^{2} \\
& + (b[t+1] - ((C_{0})  \cdot  (40) + (1 - C_{0})  \cdot  (e_{67})))^{2} \\
& + (e_{68} - ((C_{2})  \cdot  (12) + (1 - C_{2})  \cdot  (C_{24})))^{2} \\
& + (c[t+1] - ((C_{0})  \cdot  (12) + (1 - C_{0})  \cdot  (e_{68})))^{2} \\
& + (e_{72}  \cdot  (1 - e_{72}))^{2} + (e_{72}  \cdot  ((C_{1}) - (2)))^{2} \\
& + (((C_{1}) - (2))  \cdot  e_{73} - (1 - e_{72}))^{2} + (e_{74}  \cdot  (1 - e_{74}))^{2} \\
& + (e_{74}  \cdot  ((C_{24}) - (C_{14}) - (e_{76}^{2} + e_{77}^{2} + e_{78}^{2} \\
& + e_{79}^{2})))^{2} + ((1 - e_{74})  \cdot  ((C_{14}) - (C_{24}) - 1 - (e_{80}^{2} \\
& + e_{81}^{2} + e_{82}^{2} + e_{83}^{2})))^{2} + (e_{71} - (e_{72}  \cdot  e_{74}))^{2} \\
& + (e_{85} - (C_{14} + 5))^{2} + (e_{88} - (C_{14} + 5))^{2} + (e_{87} - (e_{88} - 1))^{2} \\
& + (e_{84}  \cdot  (1 - e_{84}))^{2} + (e_{84}  \cdot  ((e_{87}) - (C_{24}) - (e_{89}^{2} \\
& + e_{90}^{2} + e_{91}^{2} + e_{92}^{2})))^{2} \\
& + ((1 - e_{84})  \cdot  ((C_{24}) - (e_{87}) - 1 - (e_{93}^{2} + e_{94}^{2} + e_{95}^{2} \\
& + e_{96}^{2})))^{2} + (e_{70} - (e_{71}  \cdot  e_{84}))^{2} \\
& + (e_{99}  \cdot  (1 - e_{99}))^{2} + (e_{99}  \cdot  ((C_{1}) - (77)))^{2} \\
& + (((C_{1}) - (77))  \cdot  e_{100} - (1 - e_{99}))^{2} \\
& + (e_{101}  \cdot  (1 - e_{101}))^{2} + (e_{101}  \cdot  ((C_{24}) - (C_{3}) - (e_{103}^{2} \\
& + e_{104}^{2} + e_{105}^{2} + e_{106}^{2})))^{2} \\
& + ((1 - e_{101})  \cdot  ((C_{3}) - (C_{24}) - 1 - (e_{107}^{2} + e_{108}^{2} + e_{109}^{2} \\
& + e_{110}^{2})))^{2} + (e_{98} - (e_{99}  \cdot  e_{101}))^{2} + (e_{112} - (C_{3} + 5))^{2} \\
& + (e_{115} - (C_{3} + 5))^{2} + (e_{114} - (e_{115} - 1))^{2} \\
& + (e_{111}  \cdot  (1 - e_{111}))^{2} \\
& + (e_{111}  \cdot  ((e_{114}) - (C_{24}) - (e_{116}^{2} + e_{117}^{2} + e_{118}^{2} \\
& + e_{119}^{2})))^{2} + ((1 - e_{111})  \cdot  ((C_{24}) - (e_{114}) - 1 - (e_{120}^{2} \\
& + e_{121}^{2} + e_{122}^{2} + e_{123}^{2})))^{2} + (e_{97} - (e_{98}  \cdot  e_{111}))^{2} \\
& + (e_{69} - (e_{70} + e_{97} - e_{70}  \cdot  e_{97}))^{2} + (e_{124} - (0 - d))^{2} \\
& + (d[t+1] - ((e_{69})  \cdot  (e_{124}) + (1 - e_{69})  \cdot  (d)))^{2} \\
& + (e_{126}  \cdot  (1 - e_{126}))^{2} + (e_{126}  \cdot  ((C_{24}) - (1)))^{2} \\
& + (((C_{24}) - (1))  \cdot  e_{127} - (1 - e_{126}))^{2} \\
& + (e_{128}  \cdot  (1 - e_{128}))^{2} + (e_{128}  \cdot  ((C_{24}) - (22)))^{2} \\
& + (((C_{24}) - (22))  \cdot  e_{129} - (1 - e_{128}))^{2} + (e_{125} - (e_{126} \\
& + e_{128} - e_{126}  \cdot  e_{128}))^{2} + (e_{130} - (0 - e))^{2} \\
& + (e[t+1] - ((e_{125})  \cdot  (e_{130}) + (1 - e_{125})  \cdot  (e)))^{2} + (e_{131} - (f \\
& + 1))^{2} + (f[t+1] - ((C_{0})  \cdot  (e_{131}) + (1 - C_{0})  \cdot  (f)))^{2} \\
& + (e_{133} - (g + 1))^{2} + (e_{132} - ((C_{2})  \cdot  (e_{133}) \\
& + (1 - C_{2})  \cdot  (g)))^{2} + (g[t+1] - ((C_{0})  \cdot  (g) \\
& + (1 - C_{0})  \cdot  (e_{132})))^{2} + (p[t+1] - (C_{14}))^{2} + (q[t+1] - (C_{3}))^{2} = 0
\end{align*}
\end{document}